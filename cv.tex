\documentclass[10pt, letterpaper]{article}

% Packages:
\usepackage[
    ignoreheadfoot, % set margins without considering header and footer
    top=2 cm, % seperation between body and page edge from the top
    bottom=2 cm, % seperation between body and page edge from the bottom
    left=2 cm, % seperation between body and page edge from the left
    right=2 cm, % seperation between body and page edge from the right
    footskip=1.0 cm, % seperation between body and footer
    % showframe % for debugging
]{geometry} % for adjusting page geometry
\usepackage{titlesec} % for customizing section titles
\usepackage{tabularx} % for making tables with fixed width columns
\usepackage{array} % tabularx requires this
\usepackage[dvipsnames]{xcolor} % for coloring text
\definecolor{primaryColor}{RGB}{0, 79, 144} % define primary color
\usepackage{enumitem} % for customizing lists
\usepackage{fontawesome5} % for using icons
\usepackage{amsmath} % for math
\usepackage[
    pdftitle={Jackson Mowrys's CV},
    pdfauthor={Jackson Mowry},
    pdfcreator={LaTeX with RenderCV},
    colorlinks=true,
    urlcolor=primaryColor
]{hyperref} % for links, metadata and bookmarks
\usepackage[pscoord]{eso-pic} % for floating text on the page
\usepackage{calc} % for calculating lengths
\usepackage{bookmark} % for bookmarks
\usepackage{lastpage} % for getting the total number of pages
\usepackage{changepage} % for one column entries (adjustwidth environment)
\usepackage{paracol} % for two and three column entries
\usepackage{ifthen} % for conditional statements
\usepackage{needspace} % for avoiding page brake right after the section title
\usepackage{iftex} % check if engine is pdflatex, xetex or luatex

% Ensure that generate pdf is machine readable/ATS parsable:
\ifPDFTeX
    \input{glyphtounicode}
    \pdfgentounicode=1
    % \usepackage[T1]{fontenc} % this breaks sb2nov
    \usepackage[utf8]{inputenc}
    \usepackage{lmodern}
\fi



% Some settings:
\AtBeginEnvironment{adjustwidth}{\partopsep0pt} % remove space before adjustwidth environment
\pagestyle{empty} % no header or footer
\setcounter{secnumdepth}{0} % no section numbering
\setlength{\parindent}{0pt} % no indentation
\setlength{\topskip}{0pt} % no top skip
\setlength{\columnsep}{0cm} % set column seperation
\makeatletter
\let\ps@customFooterStyle\ps@plain % Copy the plain style to customFooterStyle
\patchcmd{\ps@customFooterStyle}{\thepage}{
    \color{gray}\textit{\small Jackson Mowry - Page \thepage{} of \pageref*{LastPage}}
}{}{} % replace number by desired string
\makeatother
\pagestyle{customFooterStyle}

\titleformat{\section}{\needspace{4\baselineskip}\bfseries\large}{}{0pt}{}[\vspace{1pt}\titlerule]

\titlespacing{\section}{
    % left space:
    -1pt
}{
    % top space:
    0.3 cm
}{
    % bottom space:
    0.2 cm
} % section title spacing

\renewcommand\labelitemi{$\circ$} % custom bullet points
\newenvironment{highlights}{
    \begin{itemize}[
        topsep=0.10 cm,
        parsep=0.10 cm,
        partopsep=0pt,
        itemsep=0pt,
        leftmargin=0.4 cm + 10pt
    ]
}{
    \end{itemize}
} % new environment for highlights

\newenvironment{highlightsforbulletentries}{
    \begin{itemize}[
        topsep=0.10 cm,
        parsep=0.10 cm,
        partopsep=0pt,
        itemsep=0pt,
        leftmargin=10pt
    ]
}{
    \end{itemize}
} % new environment for highlights for bullet entries


\newenvironment{onecolentry}{
    \begin{adjustwidth}{
        0.2 cm + 0.00001 cm
    }{
        0.2 cm + 0.00001 cm
    }
}{
    \end{adjustwidth}
} % new environment for one column entries

\newenvironment{twocolentry}[2][]{
    \onecolentry
    \def\secondColumn{#2}
    \setcolumnwidth{\fill, 4.5 cm}
    \begin{paracol}{2}
}{
    \switchcolumn \raggedleft \secondColumn
    \end{paracol}
    \endonecolentry
} % new environment for two column entries

\newenvironment{header}{
    \setlength{\topsep}{0pt}\par\kern\topsep\centering\linespread{1.5}
}{
    \par\kern\topsep
} % new environment for the header

\newcommand{\placelastupdatedtext}{% \placetextbox{<horizontal pos>}{<vertical pos>}{<stuff>}
  \AddToShipoutPictureFG*{% Add <stuff> to current page foreground
    \put(
        \LenToUnit{\paperwidth-2 cm-0.2 cm+0.05cm},
        \LenToUnit{\paperheight-1.0 cm}
    ){\vtop{{\null}\makebox[0pt][c]{
        \small\color{gray}\textit{Last updated in February 2025}\hspace{\widthof{Last updated in February 2025}}
    }}}%
  }%
}%

% save the original href command in a new command:
\let\hrefWithoutArrow\href

% new command for external links:
\renewcommand{\href}[2]{\hrefWithoutArrow{#1}{\ifthenelse{\equal{#2}{}}{ }{#2 }\raisebox{.15ex}{\footnotesize \faExternalLink*}}}


\begin{document}
    \newcommand{\AND}{\unskip
        \cleaders\copy\ANDbox\hskip\wd\ANDbox
        \ignorespaces
    }
    \newsavebox\ANDbox
    \sbox\ANDbox{}

    \placelastupdatedtext
    \begin{header}
        \textbf{\fontsize{24 pt}{24 pt}\selectfont Jackson Mowry}

        \vspace{0.3 cm}

        \normalsize
        \mbox{{\color{black}\footnotesize\faMapMarker*}\hspace*{0.13cm}Knoxville, TN}%
        \kern 0.25 cm%
        \AND%
        \kern 0.25 cm%
        \mbox{\hrefWithoutArrow{mailto:jmowry4@vols.utk.edu}{\color{black}{\footnotesize\faEnvelope[regular]}\hspace*{0.13cm}jmowry4@vols.utk.edu}}%
        \kern 0.25 cm%
        \AND%
        \kern 0.25 cm%
        \mbox{\hrefWithoutArrow{tel:+503-400-1564}{\color{black}{\footnotesize\faPhone*}\hspace*{0.13cm}503-400-1564}}%
        \kern 0.25 cm%
        \AND%
        \kern 0.25 cm%
        \mbox{\hrefWithoutArrow{https://jacksonmowry.github.io}{\color{black}{\footnotesize\faLink}\hspace*{0.13cm}jacksonmowry.github.io}}%
        \kern 0.25 cm%
        \AND%
        \kern 0.25 cm%
        \mbox{\hrefWithoutArrow{https://github.com/jacksonmowry}{\color{black}{\footnotesize\faGithub}\hspace*{0.13cm}jacksonmowry}}%
    \end{header}

    \vspace{0.3 cm - 0.3 cm}

    \section{Education}

        \begin{twocolentry}{


        \textit{Sept 2022 – Present}}
            \textbf{University of Tennessee}

            \textit{Department of Electrical Engineering and Computer Science}\\
            \textit{BS in Computer Science}


        \end{twocolentry}

        \vspace{0.10 cm}
        \begin{onecolentry}
            \begin{highlights}
                \item GPA: 3.95/4.0 (\href{https://www.eecs.utk.edu/}{eecs.utk.edu})
            \end{highlights}
        \end{onecolentry}




    \section{Experience}
        \begin{twocolentry}{
        \textit{Knoxville, TN}

        \textit{Sept 2024 - Present}}
            \textbf{Undergraduate Research Assistant}

            \textit{TENNLab}
        \end{twocolentry}

        \vspace{0.10 cm}
        \begin{onecolentry}
            \begin{highlights}
                \item Rewrote an existing neuroprocessor implementation to run on low-power RISC-V hardware, enabling a real-time demo of the DBScan algorithm, processing input from an event-based camera
                \item Developed a novel neuroprocessor to take advantage of the newly ratified RISC-V Vector 1.0 extension, enabling up to 17x better performance than contemporary solutions
                \item Currently working to develop a compiler that can take a TENNLab network and generate an all-in-one program, with the goal of running on any platform with no dependencies to facilitate researchers quickly moving from idea to prototype
            \end{highlights}
        \end{onecolentry}


        \vspace{0.2 cm}

        \begin{twocolentry}{
        \textit{Olathe, KS}

        \textit{May 2024 – Aug 2024}}
            \textbf{Software Engineer Intern}

            \textit{Garmin International}
        \end{twocolentry}

        \vspace{0.10 cm}
        \begin{onecolentry}
            \begin{highlights}
                \item Worked with an event driven architecture to enable scaling of services to meet consumer demands
                \item Developed reactive primitives enabling asynchronous queries to support highly concurrent user requests
                \item Developed integration tests to ensure consistent quality while continuously delivering updates to customers
            \end{highlights}
        \end{onecolentry}


    \section{Projects}

        \begin{twocolentry}{
        \textit{\footnotesize{github.com/jacksonmowry/vrisp}}}
        \textbf{VRISP Neuroprocessor}
        \end{twocolentry}

        \vspace{0.10 cm}
        \begin{onecolentry}
            \begin{highlights}
                \item Developed the VRISP neuroprocessor to take advantage of the newly ratified RISC-V Vector 1.0 extension, enabling up to 17x better performance than contemporary solutions while consume the same or less power
                \item Built a cross-platform code base that allows researchers to take advantage of any SIMD/Vector capabilities of their target architecture
                \item Demonstrated that networks previously deemed too complex to simulate are now possible with architectural improvements
                \item Tools Used: C++, RISC-V Vector Extension, Neuromorphic Computing
            \end{highlights}
        \end{onecolentry}


        \vspace{0.2 cm}

        \begin{twocolentry}{


        }
            \textbf{Real-time DBScan Demo}
        \end{twocolentry}

        \vspace{0.10 cm}
        \begin{onecolentry}
            \begin{highlights}
                \item Rewrote existing embedded neuromorphic simulator to improve performance by 34.4\%
                \item Reduced binary size from 150KB to 60KB through a data-oriented design approach
                \item Enabled a real-time demo of the DBScan algorithm to collaborators and other researchers
                \item Tools Used: C/C++, RISC-V, OpenCV, Bash, Event-based Cameras, Neuromorphic Computing
            \end{highlights}
        \end{onecolentry}

    \section{Coursework}
        \textbf{Spring 2025:}
            \begin{highlights}
                \item COSC 593 Embedded Neuromorphic Systems
                \item COSC 366 Introduction to Cybersecurity
                \item COSC 365 Programming Languages and Systems
                \item COSC 361 Operating Systems
            \end{highlights}
        \textbf{Fall 2024:}
            \begin{highlights}
                \item COSC 360 Systems Programming
                \item COSC 312 Algorithm Analysis and Automata
                \item COSC 340 Software Engineering
                \item MATH 251 Matrix Algebra I
            \end{highlights}
        \textbf{Spring 2024:}
            \begin{highlights}
                \item COSC 307 Honors: Data Structures and Algorithms II
                \item COSC 311 Discrete Structures
                \item ECE  313 Probability and Random Variables
                \item PHYS 136 Introduction to Physics for Math Majors II
            \end{highlights}
        \textbf{Fall 2023:}
            \begin{highlights}
                \item COSC 202 Data Structures and Algorithms I
                \item MATH 148 Honors: Calculus II
                \item PHYS 136 Introduction to Physics for Math Majors I
            \end{highlights}
        \textbf{Spring 2022 - Spring 2023:}
            \begin{highlights}
                \item COSC 230 Computer Organization
                \item COSC 102 Introduction to Computer Science
                \item COSC 101 Introduction to Programming
            \end{highlights}

    \section{Technologies}




        \begin{onecolentry}
            \textbf{Languages:} C++, C, Go, V, Bash, SQL, Elixir, Lisp, Java
        \end{onecolentry}

        \vspace{0.2 cm}

        \begin{onecolentry}
            \textbf{Technologies:} Linux, Git, Make, Caddy, Emacs, Vim
        \end{onecolentry}




\end{document}
