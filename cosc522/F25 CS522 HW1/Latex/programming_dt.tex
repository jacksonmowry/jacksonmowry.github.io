\section{Programming (55 points)}

In this assignment, you will implement a decision tree learning algorithm for binary classification. In addition, we will ask you to run some end-to-end experiments on two tasks (predicting whether or not a patient has heart disease / predicting the final grade for high school students) and report your results. Please confine all code you write to a single, self-contained file titled \texttt{decision\_tree.py}. 


\subsection{The Tasks and Datasets}
\label{sec:data}

\paragraph{Materials} Download the zip file from the course website. The zip file will have a handout folder that contains all the data that you will need in order to complete this assignment.


\paragraph{Datasets}

The handout contains three datasets. Each one contains attributes and labels and is already split into training and validation data. The first line of each \lstinline{.tsv} file contains the name of each attribute, and \emph{the class label is always the last column}.

\begin{enumerate}
\item \textbf{heart:}
    The first task is to predict whether a patient has been (or will be) diagnosed with heart disease, based on available patient information. The attributes (aka. features) are: 
    \begin{enumerate}
        \item \lstinline{sex}: The sex of the patient---1 if the patient is male, and 0 if the patient is female.
        \item \lstinline{chest_pain}: 1 if the patient has chest pain, and 0 otherwise.
        \item \lstinline{high_blood_sugar}: 1 if the patient has high blood sugar ($>$120 mg/dl fasting), and 0 otherwise.
        \item \lstinline{abnormal_ecg}: 1 if the patient had an abnormal resting electrocardiographic (ECG) reading, and 0 otherwise. 
        \item \lstinline{angina}: 1 if exercise induced angina in the patient, and 0 otherwise. Angina is a type of severe chest pain.
        \item \lstinline{flat_ST}: 1 if the patient's ST segment (a section of an ECG) was flat during exercise, and 0 if it had some slope.
        \item \lstinline{fluoroscopy}: 1 if a physician used fluoroscopy, and 0 otherwise. Fluoroscopy is an imaging technique used to see the flow of blood through the heart.
        \item \lstinline{thalassemia}: 1 if the patient is known to have thalassemia, and 0 otherwise. Thalassemia is a blood disorder that may impair the oxygen-carrying capacity of the patient's red blood cells.
        \item \lstinline{heart_disease}: 1 if the patient was diagnosed with heart disease, and 0 otherwise. This is the class label you should predict.
    \end{enumerate}
    The training data is in \lstinline{heart_train.tsv}, and the validation data in \lstinline{heart_valid.tsv}.
\item \textbf{education:}
    The second task is to predict the final grade for high school students. The attributes are student grades on 5 multiple choice assignments \emph{M1} through \emph{M5}, 4 programming assignments \emph{P1} through \emph{P4}, and the final exam \emph{F}. Values of 1 indicate that a student received an A, and 0 indicates that the student did not receive an A. The training data is in \newline \lstinline{education_train.tsv}, and the validation data in \lstinline{education_valid.tsv}.
\item \textbf{small:}
    We also include \lstinline{small_train.tsv} and \lstinline{small_valid.tsv}---a small, purely for demonstration version of the \textbf{heart} dataset, with \emph{only} attributes \lstinline{chest_pain} and \lstinline{thalassemia}.  

\end{enumerate}



\newpage
\subsection{Program: Decision Tree}
\label{sec:decisiontree}

Your code should learn a decision tree with a specified maximum depth, print the decision tree in a specified format, predict the labels of the training and validation examples, and calculate training and validation errors.


\textbf{Your implementation must satisfy the following requirements:}
\begin{itemize}
\item Use mutual information to determine which attribute to split on.
\item Be sure you’re correctly weighting your calculation of mutual information. For a split on attribute X, $I(Y;X) = H(Y)-H(Y|X) = H(Y) - P(X=0)H(Y|X = 0) - P(X = 1)H(Y|X = 1)$.
\item As a stopping rule, only split on an attribute if the mutual information is $>$ certain threshold. By default, you should set the threshold to $0$. Certain questions in \ref{sec:empirical questions} may require you to experiment with different threshold.
\item Do not grow the tree beyond a certain max-depth, if specified in the question. For example, for a maximum depth of 3, split a node only if the mutual information is $>$ 0 and the current level of the node is $< 3$.
\item Use a majority vote of the labels at each leaf to make classification decisions. If the vote is tied, choose the label that is numerically larger (i.e., $1$ should be chosen before $0$)
\item It is possible for different columns to have equal values for mutual information. In this case, you should split on the \emph{\textbf{first column to break ties}} (e.g. if column 0 and column 4 have the same mutual information, use column 0).
\end{itemize}

\subsubsection{Getting Started}
\label{sec:getting started}

Careful planning will help you to correctly and concisely implement your decision tree learning algorithm. Here are a few \emph{hints} to get you started:
\begin{itemize}
    \item Write helper functions to calculate entropy and mutual information.
    \item It is best to think of a decision tree as a collection of nodes, where nodes are either leaf nodes (where final decisions are made) or interior nodes (where we split on attributes). It is helpful to design a function to train a single node (i.e. a depth-0 tree), and then recursively call that function to create sub-trees.
    \item In the recursion, keep track of the depth of the current tree so you can stop growing the tree beyond the max-depth.
    \item Implement a function that takes a learned decision tree and data as inputs, and generates predicted labels. You can write a separate function to calculate the error of the predicted labels with respect to the given (ground-truth) labels.
    \item Be sure to correctly handle the case where the specified maximum depth is greater than the total number of attributes.
    \item Be sure to handle the case where max-depth is zero (i.e. a majority vote classifier). 

\end{itemize}


\clearpage

\subsubsection{Output: Printing the Tree}
\label{sec:printtree}

You should also write a function to pretty-print your learned decision tree. \textbf{Your function should print your tree only \emph{after} you are done generating the fully-trained tree.} Each row should correspond to a node in the tree. They should be printed using a \emph{pre-order depth-first-search} traversal (but you may print left-to-right or right-to-left, i.e. your answer does not need to have exactly the same order as the reference below). Print the \texttt{attribute} of the node's parent (e.g., \texttt{chest\_-pain} in example format below) and the \texttt{attribute value} corresponding to the node (e.g., value of 0 for \texttt{chest\_-pain}). Also include the sufficient statistics (i.e. count of negative (label as 0) / positive examples (label as 1)) for the data passed to that node. The row for the root should include \emph{only} those sufficient statistics. A node at depth $d$, should be prefixed by $d$ copies of the string `$\mid$ '.

Below, we have provided the recommended format for printing the tree (the \texttt{2} in the python command below denotes the maximum tree depth). You can print it directly rather than to a file. 

\begin{lstlisting}[language=Shell]
$ python decision_tree.py small_train.tsv small_valid.tsv 2

[14 0/14 1]
| chest_pain = 0: [4 0/12 1]
| | thalassemia = 0: [3 0/4 1]
| | thalassemia = 1: [1 0/8 1]
| chest_pain = 1: [10 0/2 1]
| | thalassemia = 0: [7 0/0 1]
| | thalassemia = 1: [3 0/2 1]
\end{lstlisting}


However, you should be careful that the tree might not be full. For example, with a different subset of the small dataset, there may be no nodes under \lstinline{chest_pain = 0} if all labels are the same.

The following pretty-print shows the education dataset with max-depth 3.  Use this example to check your code before submitting your pretty-print of the heart dataset.

\begin{lstlisting}[language=Shell]
$ python decision_tree.py education_train.tsv education_valid.tsv 3

[65 0/135 1]
| F = 0: [42 0/16 1]
| | M2 = 0: [27 0/3 1]
| | | M4 = 0: [22 0/0 1]
| | | M4 = 1: [5 0/3 1]
| | M2 = 1: [15 0/13 1]
| | | M4 = 0: [14 0/7 1]
| | | M4 = 1: [1 0/6 1]
| F = 1: [23 0/119 1]
| | M4 = 0: [21 0/63 1]
| | | M2 = 0: [18 0/26 1]
| | | M2 = 1: [3 0/37 1]
| | M4 = 1: [2 0/56 1]
| | | P1 = 0: [2 0/15 1]
| | | P1 = 1: [0 0/41 1]
\end{lstlisting}

The numbers in brackets give the number of positive and negative labels from the training data in that part of the tree.

\begin{notebox}
At this point, you should be able to answer questions 1-5 in the ``Empirical Questions" of this handout.  Write your solutions in the template provided. 
\end{notebox}


\subsection{Written: Empirical Questions}
\label{sec:empirical questions}

\begin{enumerate}
    \item {\bf [5 Points]} Report the validation accuracy of the decision tree classifiers trained on the three included datasets (with no limitation on maximum depth and all configurations like information gain threshold etc. set to the default value):

        \begin{itemize}
            \item \texttt{small} dataset: 
            \begin{tcolorbox}[fit,height=0.5cm, width=3cm, blank, borderline={1pt}{-2pt},nobeforeafter]
                % Your solution here
            \end{tcolorbox}

            
            \item \texttt{heart} dataset:
            \begin{tcolorbox}[fit,height=0.5cm, width=3cm, blank, borderline={1pt}{-2pt},nobeforeafter]
                % Your solution here
            \end{tcolorbox}

            
            \item \texttt{education} dataset: 
            \begin{tcolorbox}[fit,height=0.5cm, width=3cm, blank, borderline={1pt}{-2pt},nobeforeafter]
                % Your solution here
            \end{tcolorbox}
        \end{itemize}

    \item {\bf [5 Points]} In the box below, paste the decision tree learned on \texttt{heart\_train.tsv} under max depth limitation of $4$. Be sure to follow the formatting requirement in section \ref{sec:printtree}.

    \begin{tcolorbox}[fit,height=17cm, width=15cm, blank, borderline={1pt}{-2pt},nobeforeafter]
    
    \end{tcolorbox}
    

\item {\bf [10 Points]} Now use \texttt{heart\_val.tsv} to perform reduced error pruning on the tree you learned in the previous question; break ties in favor of shorter trees. In the box below, paste the decision tree learned after pruning. Be sure to follow the formatting requirement in section \ref{sec:printtree}.

\textbf{Note: }After performing reduced error pruning, this is what the education dataset with max-depth 3 looks like: 
\begin{lstlisting}[language=Shell]
[65 0/135 1]
| F = 0:  [42 0/16 1]
| | M2 = 0:  [27 0/3 1]
| | M2 = 1:  [15 0/13 1]
| | | M4 = 0:  [14 0/7 1]
| | | M4 = 1:  [1 0/6 1]
| F = 1:  [23 0/119 1]
\end{lstlisting}

    \begin{tcolorbox}[fit,height=10cm, width=15cm, blank, borderline={1pt}{-2pt},nobeforeafter]
    \end{tcolorbox}
    
    
\end{enumerate}
\newpage
In the following questions, we will explore the performance of learned decision trees on validation data and introduce the problem of ``overfitting''. \textbf{All questions below are asked on the \texttt{heart} dataset (\texttt{heart\_train.tsv} and \texttt{heart\_val.tsv})}.

\begin{enumerate}
    \item[4.] {\bf [15 Points]} Iterate the maximum depth limitation in the range $[0, 8]$, collect the training accuracy and validation accuracy. Plot both metrics with respect to the maximum depth in a single figure. Label your axes as well as your training and validation accuracy plots.

    Is the validation accuracy monotonically increasing as the maximum depth limit increases? What about the training accuracy? Report the maximum validation accuracy observed and the maximum depth limit that at which it is achieved.

    \begin{tcolorbox}[fit,height=7cm, width=15cm, blank, borderline={1pt}{-2pt},nobeforeafter]
        % Your solution here
        
        
    \end{tcolorbox}


    \end{enumerate}


    The phenomenon you have (hopefully) witnessed, wherein the model struggles to extrapolate effectively from the training dataset to the validation dataset, is widely referred to as \textit{overfitting}. One approach to mitigating the overfitting dilemma involves constraining the complexity of the classifier, and this can be done during the construction of the decision tree (recall the diverse \texttt{Base Conditions} in the lecture slides).

    Constraining the maximum depth of the decision tree is perhaps the most straightforward method for addressing the issue of overfitting but there are many alternatives. 


    \begin{enumerate}
    \item[5.] {\bf [10 Points]} Among these alternatives is the imposition of a restriction on the size of splitting nodes. Specifically, if a node comprises \textit{fewer than} $M$ data points during the training process, further splitting is prohibitted.

    Train a decision tree under different minimum splitting sizes, with $C$ ranging from $0$ to the number of data points in the \texttt{heart} dataset: what is the optimal minimum splitting size, and what is the validation accuracy under this value?

    \textbf{Note}: If there are multiple minimum splitting sizes that achieve the optimal validation accuracy, report the smallest of them.
    
    \begin{tcolorbox}[fit,height=1.5cm, width=15cm, blank, borderline={1pt}{-2pt},nobeforeafter]
        % Your solution here
    \end{tcolorbox}
    

\newpage
    \item[6.] {\bf [10 Points]} Another technique for mitigating overfitting involves changing the mutual information threshold. In this approach, at each node, if the best possible mutual information from a subsequent split falls below a designated threshold $\tau$, the node is not divided any further.

    Train a decision tree with mutual information thresholds of $\{0.00, 0.01, \cdots, 0.99, 1.00\}$: what is the optimal information gain threshold, and what is the validation accuracy under this value? 
    
    \textbf{Note}: If there are multiple mutual information thresholds that achieve the optimal validation accuracy, report the smallest of them.
    
    \begin{tcolorbox}[fit,height=1.5cm, width=15cm, blank, borderline={1pt}{-2pt},nobeforeafter]
        % Your solution here
    \end{tcolorbox}


    
\end{enumerate}

\subsection{Submission Instructions}

\paragraph{Programming}
Please ensure you have completed the following file(s) for submission.

\begin{verbatim}
decision_tree.py
\end{verbatim}

When submitting your solution, make sure to select and upload the file(s) shown above. \textbf{Any other files will be deleted.} Ensure the files have the exact same spelling and letter casing as above. We may manually grade your code for the purposes of assigning partial credit if your solutions to the empirical questions differ from our reference solutions. 

\paragraph{Written Questions}
Make sure you have completed all questions from Written component (including the collaboration policy questions) in the template provided.  When you have done so, please submit your document in \textbf{PDF format} along with the code as a single Zip file to the corresponding assignment slot on Canvas.

