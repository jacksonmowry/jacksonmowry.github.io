% Created 2025-09-03 Wed 09:33
% Intended LaTeX compiler: pdflatex
\documentclass[11pt]{article}
\usepackage[utf8]{inputenc}
\usepackage[T1]{fontenc}
\usepackage{graphicx}
\usepackage{longtable}
\usepackage{wrapfig}
\usepackage{rotating}
\usepackage[normalem]{ulem}
\usepackage{amsmath}
\usepackage{amssymb}
\usepackage{capt-of}
\usepackage{hyperref}
\author{Jackson Mowry}
\date{Wed Sep  3 09:33:29 2025}
\title{Hw3}
\hypersetup{
 pdfauthor={Jackson Mowry},
 pdftitle={Hw3},
 pdfkeywords={},
 pdfsubject={},
 pdfcreator={Emacs 30.2 (Org mode 9.7.31)}, 
 pdflang={English}}
\begin{document}

\maketitle
\begin{enumerate}
\item Pseudocode for the padding function described in Section 5.1.1.
\end{enumerate}
\begin{verbatim}
# Equation we're trying to solve l + 1 + k == 448mod512
# Simplifies to 448 - (l + 1) = k

def pad_message(message: u1[]) -> void:
    message_len: u64 = length of message in bits
    k = 448 - (message_len + 1)

    message += 1
    for 0 to k:
        message += 0
    end

    message += message_len as u1[]

\end{verbatim}

\begin{enumerate}
\item Pseudocode for calculating the messaging schedule (W\textsubscript{t}) described in Section 6.1.2.
\end{enumerate}
\begin{verbatim}
def message_schedule(message: u1[], block: u32) -> u32[]:
    current_block = message[(block * 512)..(block * 512 + 512)]
    w: u32[] = []

    for i = 0 to 80:
        if i >= 0 && i <= 15:
            w += current_block[(i * 32)..(i * 32 + 32)]
        else
            w += rotl(w[i-3] ^ w[i-8] ^ w[t-14] ^ w[t-16])
        fi
    end

    return w
\end{verbatim}
\end{document}
