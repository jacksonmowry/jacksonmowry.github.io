% Created 2024-09-06 Fri 09:36
% Intended LaTeX compiler: pdflatex
\documentclass[11pt]{article}
\usepackage[utf8]{inputenc}
\usepackage[T1]{fontenc}
\usepackage{graphicx}
\usepackage{longtable}
\usepackage{wrapfig}
\usepackage{rotating}
\usepackage[normalem]{ulem}
\usepackage{amsmath}
\usepackage{amssymb}
\usepackage{capt-of}
\usepackage{hyperref}
\date{\today}
\title{One Five}
\hypersetup{
 pdfauthor={},
 pdftitle={One Five},
 pdfkeywords={},
 pdfsubject={},
 pdfcreator={Emacs 29.4 (Org mode 9.8)}, 
 pdflang={English}}
\begin{document}

\maketitle
\tableofcontents

\section{Recall}
\label{sec:org78783c5}
Multiply a row by a nonzero constant c
Interchange 2 rows
Add a constant c times one row to another
\section{Definition 1}
\label{sec:orge89e306}
Matricies A and B are said to be row equivalent if either (hence each) can be obtained from the other by a sequence of elementary row operations
\section{Definition 2}
\label{sec:org5d8fb5d}
A matrix E is called an elementary matrix if it can be obtained from an identity matrix by performing a single elementary row operation
\subsection{Example 1}
\label{sec:org6542267}
Listed below are four elementary matricies and the oerations that product them

\begin{bmatrix}1 & 0 \\ 0 & -3\end{bmatrix} \begin{bmatrix}1&0&0&0\\0&0&0&1\\0&0&1&0\\0&1&0&0\end{bmatrix} \begin{bmatrix}1&0&3\\0&1&0\\0&0&1\end{bmatrix} \begin{bmatrix}1&0&0\\0&1&0\\0&0&1\end{bmatrix}

\begin{enumerate}
\item Multiply the second row of I\textsubscript{2} by -3
\item Interchange the second and fourth rows of I\textsubscript{4}
\item Add 3 times the third row of I\textsubscript{3} to the first row
\item Multiply the first row of I\textsubscript{3} by 1
\end{enumerate}


Every elementary matrix is invertible, and the inverse is also an elementary matrix
\begin{itemize}
\item any elementary row operation we want to do on Matrix A, we could do it on the identity matrix, and left multiply it to matrix A to get what we want
\end{itemize}

For example, given \(\begin{bmatrix}1&2&3\\4&5&6\\7&8&9\end{bmatrix}\)

Find the elementary matrix E, so that
\begin{enumerate}
\item EA = \(\begin{bmatrix}3 & 6 & 9 \\ 4 & 5 & 6 \\ 7 & 8 & 9\end{bmatrix}\)
\item EA = \(\begin{bmatrix}7&8&9\\4&5&6\\1&2&3\end{bmatrix}\)
\item EA = \(\begin{bmatrix}1&2&3\\4&5&6\\8&9&10\end{bmatrix}\)
\end{enumerate}

\begin{verbatim}
a = [1,0,0;0,1,0;1,0,1]
b = [1,2,3;4,5,6;7,8,9]

a * b
\end{verbatim}
\subsection{Theorem}
\label{sec:org7f5ccfd}
If a is an n x n matrix, then the following statements are equivlant, that is all true or all false
\begin{itemize}
\item A is invertible
\item Ax = o has only the trivial solution
\item The reduced row echelon form of A is I\textsubscript{n}
\item A is expressible as a product of elementary matricies
\end{itemize}
\section{A Method for Inverting Matricies}
\label{sec:org57c8d75}
To find the inverse of an invertible matrix A, find a sequence of elementary row operations that reduces A to the identity and then perform that same sequence of operations on I\textsubscript{n} to obtain A\textsuperscript{-1}
\subsection{Procedure}
\label{sec:orga5baaf9}
We want to reduce A to the identity matrix by row operations and simultaneously apply these operations to I to produce A\textsuperscript{-1}. To accomplist this we will adjoin the identity matrix to the right side of A, thereby producing a partitioned matrix of the form
\[
[A|I]
\]

Then we will apply row operations to this matrix until the left side is reduced to I; these operations will convert the right side to A\textsuperscript{-1}, so the final matrix will have the form
\[
[I|A^{-1}]
\]
\subsection{Example 3}
\label{sec:orgc5e1161}
Find the inverse of A if it exists
\begin{enumerate}
\item A = \(\begin{bmatrix}1&2&3\\2&5&3\\1&0&8 \| 1&2&3\end{bmatrix}\)
\$\begin{bmatrix}\end{bmatrix}
\$\begin{bmatrix}\end{bmatrix}
\$\begin{bmatrix}\end{bmatrix}
\$\begin{bmatrix}\end{bmatrix}
\$\begin{bmatrix}\end{bmatrix}
\$\begin{bmatrix}\end{bmatrix}
\$\begin{bmatrix}\end{bmatrix}
\$\begin{bmatrix}\end{bmatrix}
\item A = \(\begin{bmatrix}1&6&4\\2&4&-1\\-1&2&5\end{bmatrix}\)
\end{enumerate}
\end{document}
