% Created 2024-09-04 Wed 09:41
% Intended LaTeX compiler: pdflatex
\documentclass[11pt]{article}
\usepackage[utf8]{inputenc}
\usepackage[T1]{fontenc}
\usepackage{graphicx}
\usepackage{longtable}
\usepackage{wrapfig}
\usepackage{rotating}
\usepackage[normalem]{ulem}
\usepackage{amsmath}
\usepackage{amssymb}
\usepackage{capt-of}
\usepackage{hyperref}
\date{\today}
\title{One Three}
\hypersetup{
 pdfauthor={},
 pdftitle={One Three},
 pdfkeywords={},
 pdfsubject={},
 pdfcreator={Emacs 29.4 (Org mode 9.8)}, 
 pdflang={English}}
\begin{document}

\maketitle
\tableofcontents

\section{Objectives}
\label{sec:orge602a99}
\begin{itemize}
\item Add and subtract matricies
\item Multiply matricies
\item Writing a system of linear equations as a matrix product
\item Write a matrxi product as the linear combination of the columns
\item Compute the tract and transpost of a matrix
\end{itemize}
\section{Consistent Factors}
\label{sec:org2d65353}
A matrix with \emph{m} rows and \emph{n} columns is called an \emph{m} by \emph{n} matrix
Every location in a matrix A is double subscripted
The a\textsubscript{i,j}th number is found in the /i/th row and /j/th column
3 by 4 matrix
5, 9, 3
Row matricies have one row for example, column matricies have one column, also called vector
\section{Matrix Arithmetic}
\label{sec:org5039691}
\subsection{Defintion}
\label{sec:orgfaa89d4}
Two matricies A and B are equal, if they have the same dimension (m x n), and their corresponding entries are identical. (A)\textsubscript{ij} = = a\textsubscript{ij} = b\textsubscript{ij} = (B)\textsubscript{ij} for 1 <= \emph{i} <= m, and 1 <= \emph{j} <= \emph{n}
\subsection{Addition/Subtraction}
\label{sec:org4606370}
Two matricies have a sumer or different if and only if they have the same dimensions (m x n), and their sum is defined to be (A + B)\textsubscript{ij} = a\textsubscript{ij} + b\textsubscript{ij}

\begin{center}
\begin{tabular}{rrr}
1 & 2 & 4\\
1 & 3 & 9\\
\end{tabular}
\end{center}
\begin{itemize}
\item 
\end{itemize}
\begin{center}
\begin{tabular}{rrr}
2 & 3 & 6\\
4 & 5 & 20\\
\end{tabular}
\end{center}
=
\begin{center}
\begin{tabular}{rrr}
3 & 5 & 10\\
5 & 8 & 29\\
\end{tabular}
\end{center}

\begin{center}
\begin{tabular}{rr}
1 & 3\\
4 & 7\\
2 & -1\\
\end{tabular}
\end{center}
\begin{itemize}
\item 
\end{itemize}
\begin{center}
\begin{tabular}{rr}
4 & 2\\
1 & 3\\
0 & 6\\
\end{tabular}
\end{center}
=
\begin{center}
\begin{tabular}{rr}
(-3) & 1\\
3 & 4\\
2 & -7\\
\end{tabular}
\end{center}
\subsection{Definition}
\label{sec:org4b84108}
If A is an m x n matrix and c is a scalar, then the \textbf{scalar multiple} of A by c is the m x n matric given by cA = (ca\textsubscript{ij})

\begin{enumerate}
\item 
\end{enumerate}
\begin{center}
\begin{tabular}{rr}
7 & 63\\
56 & -14\\
\end{tabular}
\end{center}
\begin{enumerate}
\item 
\end{enumerate}
\begin{center}
\begin{tabular}{lll}
ka & kb & kc\\
kd & ke & kf\\
\end{tabular}
\end{center}
\begin{enumerate}
\item 
\end{enumerate}
\begin{center}
\begin{tabular}{rr}
0 & 0\\
0 & 0\\
0 & 0\\
\end{tabular}
\end{center}
\begin{enumerate}
\item 
\end{enumerate}
\begin{center}
\begin{tabular}{rr}
16 & 5\\
2 & 13\\
\end{tabular}
\end{center}
\subsection{Definition}
\label{sec:org3f7c228}
Two matricies A and B can be multiplied if and only if the number of columns of A matches the number of rows of B

A\textsubscript{m x n}B\textsubscript{n x p} = C\textsubscript{m x p}
and (AB)\textsubscript{ij} = \Sum{k=1}{n}a\textsubscript{ik}b\textsubscript{kj} = a\textsubscript{i1}b\textsubscript{1j} + a\textsubscript{i2}b\textsubscript{2j} + a\textsubscript{i3}b\textsubscript{3j} + \ldots{} + a\textsubscript{in}b\textsubscript{nj}

If A is 3x4 and B is 4x5 => 3x5
If A is 3x4 and B is 3x4 => invalid multiplication
\subsubsection{How-to Matrix Multiply}
\label{sec:org173f627}
Row \emph{i} times column \emph{j}
\begin{enumerate}
\item 1
\label{sec:org4189598}
\begin{verbatim}
a = [1,3]
b = [a,b;c,d]
a * b
\end{verbatim}

row1 = [1,3] \& column1 = [a;c]
turns into
1(a) + 3(c) = a+3c

row1 = [1,3] \& column2 = [b;d]
turns into
1(b) + 3(d) = b+3d

\begin{center}
\begin{tabular}{ll}
row1*colum1 & row1*column2\\
\hline
a+3c & b+3d\\
\end{tabular}
\end{center}
\item 2
\label{sec:org7dcc243}
\begin{verbatim}
a = [0,-1;1,0]
b = [5;1]

a * b
\end{verbatim}

\begin{center}
\begin{tabular}{r}
0(5)+(-1)1\\
1(5)+0(1)\\
\end{tabular}
\end{center}

\begin{center}
\begin{tabular}{r}
(-1)\\
5\\
\\
\end{tabular}
\end{center}
\item 3
\label{sec:org45c19b1}
\begin{verbatim}
a = [1,2;3,4;5,6]
b = [4,3;5,1]

a * b
\end{verbatim}

\begin{center}
\begin{tabular}{rr}
1(4)+2(5) & 1(3)+2(1)\\
3(4)+4(5) & 3(3)+4(1)\\
5(4)+6(5) & 5(3)+6(1)\\
\end{tabular}
\end{center}

\begin{center}
\begin{tabular}{rr}
14 & 5\\
32 & 13\\
50 & 21\\
\end{tabular}
\end{center}
\item 4
\label{sec:org7e33447}
\begin{center}
\begin{tabular}{ll}
a\textsubscript{11} & a\textsubscript{12}\\
a\textsubscript{21} & a\textsubscript{22}\\
\end{tabular}
\end{center}
and
\begin{center}
\begin{tabular}{ll}
b\textsubscript{11} & b\textsubscript{12}\\
b\textsubscript{21} & b\textsubscript{22}\\
\end{tabular}
\end{center}

\begin{center}
\begin{tabular}{ll}
a\textsubscript{11}(b\textsubscript{11}) + a\textsubscript{22}(b\textsubscript{21}) & a\textsubscript{11}(b\textsubscript{12}) + a\textsubscript{12}(b\textsubscript{22})\\
a\textsubscript{21}(b\textsubscript{11}) + a\textsubscript{22}(b\textsubscript{21}) & a\textsubscript{21}(b\textsubscript{12}) + a\textsubscript{22}(b\textsubscript{22})\\
\end{tabular}
\end{center}
\item 5
\label{sec:org04bb211}
\begin{center}
\begin{tabular}{rr}
1(5)+2(6) & 1(-1)+2(7)\\
3(5)+4(6) & 3(-1)+4(7)\\
\end{tabular}
\end{center}

\begin{center}
\begin{tabular}{rr}
17 & 13\\
39 & 25\\
\end{tabular}
\end{center}
\item 6
\label{sec:org40fd589}
\begin{center}
\begin{tabular}{rr}
5(1)+-1(3) & 5(2)+-1(4)\\
6(1)+7(3) & 6(2)+7(4)\\
\end{tabular}
\end{center}

\begin{center}
\begin{tabular}{rr}
2 & 6\\
27 & 40\\
\end{tabular}
\end{center}
\item 7
\label{sec:org6435528}
\begin{center}
\begin{tabular}{rr}
1(1)+3(4)+5(6) & 1(6)+3(1)+5-1)\\
\end{tabular}
\end{center}
\end{enumerate}
\section{Coefficient Matrix}
\label{sec:orged86010}
{[}A|b] where A are the left hand side, b is the right hand side of the equals sign
\section{Linear Combination}
\label{sec:orgd5d5e0b}
Matrix product as linear combination
\subsection{Theorem}
\label{sec:org64499e3}
If A\textsubscript{m x n} and x\textsubscript{n x 1}, then Ax can be expressed as a linear combination of the column vectors of A in which coefficients are the entires of X(vector)

In general: A\textsubscript{x} = [column 1]x\textsubscript{1} + [column 2]x\textsubscript{2} + \ldots{} + [column n]x\textsubscript{n}
\section{Transpose a Matrix}
\label{sec:orgde0a5af}
If any m x n matrix is transposed then it's dimensions will become n x m, row of A becomes column of A\textsubscript{T}

\[(A^{T})_{ij} = (A)_{ji}\]
\section{Trace of a Matrix}
\label{sec:orgbd13c63}
If A is a square matrix, then the tract of A, denoted by \textbf{tr(A)} is defined to be the sum of the entries on the main diagonal of \textbf{A}. The trace of A is undefined if A is not a square matrix.
\section{Matrix Polynomials}
\label{sec:org1ec76fd}
Given \(2x^{2}-3x+4\) find f(a)
We need to add an 'I' that is the same dimensions as the matrix, in this case 4I\textsubscript{2}
\begin{verbatim}
A = [4,1;3,5]

2 * (A * A) - (3*A) + 4*eye(2)

2*[19,9;27,28]-[12,3;9,15]+[4,0;0,4]
\end{verbatim}
\section{The Inverse of a Square Matrix}
\label{sec:orga5c9bd0}
Any nxn matrix A is said to be invertible, or nonsingular, if there exists a matrix B such that AB = I = Ba.
The inverse is written as A\textsuperscript{-1}
If A has no inverse it is said to be singular

Need to check if AB = I \&\& BA = I

If you multiply a matrix and its inverse you should get the identity matrix of the same size
\subsection{Given A = [5,1;4,1] and B = [1,-1;-4,5], check if A and B are inverses}
\label{sec:org00c6f79}
We take [a,b;c,d] -> [d,-b;-c,a]

\begin{verbatim}
a = [5,1;4,1]
b = [1,-1;-4,5]

b*a
\end{verbatim}
\subsection{Given A = [1,3;1,6] and B = [2,-1;-1/3,1/3] check if A and B are inverse}
\label{sec:orge9c67b9}
\begin{verbatim}
A = [1,3;1,6]
B = [2,-1;-1/3,1/3]

A*B
\end{verbatim}
\subsection{Theorem}
\label{sec:org75aeb11}
A matrix A is only invertible if \(ad-bc \neq 0\), in which case the inverse is given by the formula

\[
A^{-1} = \frac{1}{ad-bc}\begin{bmatrix}d & -b \\ -c & a\end{bmatrix}
\]

\(ad-bc\) is called the determinant of the matrix A (cross multiply)
\subsection{Practice}
\label{sec:orgcfc4aa4}
\begin{verbatim}
a = [1,2;3,4]
b = [-1,2;3,-2]
c = [2,-1;-4,2]

inv(a)
\end{verbatim}
\subsection{Why Inverse Matrix?}
\label{sec:orgd08b4f3}
Helps us to solve a linear system

Given a matrix equation A \(\rightarrow\)\textsubscript{x} = b\textsuperscript{\(\rightarrow\)}, we could solve the equation by applying A\^{}\textsubscript{-1} to both sides to get
\[
A^{-1}Ax^{\rightarrow{}} = A^{-1}b^{\rightarrow}
\]
\[
x^{\rightarrow} = A^{-1}b^{\rightarrow{}}
\]

Note, if A\textsuperscript{-1} does not exist, the the equation has no solutions
\subsubsection{Examples}
\label{sec:org5c24c4a}
Solve the system by matrix inversion

\[
x_{1}+2x_{2} = 4
3x_{1}+4x_{2} = 10
\]

$\backslash$[
\begin{bmatrix}1 & 2 \\ 3 & 4\end{bmatrix} \begin{bmatrix}x_{1}\\x_{2}\end{bmatrix} = \begin{bmatrix}4\\10\end{bmatrix}
$\backslash$]

\[
A^{-1} = \begin{bmatrix}-2&1\\\frac{3}{2}&\frac{-1}{2}\end{bmatrix}
\]
\subsection{Properties of Inverses}
\label{sec:org3df6d87}
\end{document}
