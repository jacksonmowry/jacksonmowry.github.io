% Created 2024-10-18 Fri 09:17
% Intended LaTeX compiler: pdflatex
\documentclass[11pt]{article}
\usepackage[utf8]{inputenc}
\usepackage[T1]{fontenc}
\usepackage{graphicx}
\usepackage{longtable}
\usepackage{wrapfig}
\usepackage{rotating}
\usepackage[normalem]{ulem}
\usepackage{amsmath}
\usepackage{amssymb}
\usepackage{capt-of}
\usepackage{hyperref}
\date{\today}
\title{One Nine}
\hypersetup{
 pdfauthor={},
 pdftitle={One Nine},
 pdfkeywords={},
 pdfsubject={},
 pdfcreator={Emacs 29.4 (Org mode 9.7.11)}, 
 pdflang={English}}
\begin{document}

\maketitle
\tableofcontents

\section{Recall}
\label{sec:orgf90b711}
College algebra
\(f \circ{} g(x) = f(g(x))\)
\section{Objectives}
\label{sec:orgc3c7e9f}
Compose linear transformations
Compute the compositions of reflection and rotation transformations
Compute the inverse of a transformation
\section{Notes}
\label{sec:orgb683e36}
Given a linear transofmration \(T{A}: \mathbb{R}^{n} \rightarrow{} \mathhbb{m}\) and \(T_{B} : \mathbb{R}^{m} \rightarrow{} \mathbb{R}^{k}\)  define the composition \((T_{B}\circ{}T_{A})(x)\) to be the function \(T_{B}(T{A}(x))\)
\begin{itemize}
\item In fact \(T_{B}\circ{}T_{A} = T_{BA}\)
\end{itemize}
\section{Examples}
\label{sec:orgbfbb7d1}
find the standard matrix for each of the following transformation, and find the standard matrix for the compositie functions \(T_{1}\circ{}T_{2}\) and \(T_{2}\circ{}T_{1}\)

\[
T_{1}(e_{1}) = T_{1}(1,0) = (1+0,1-0) = (1,1)
\]
\[
T_{1}(e_{2}) = T_{1}(0,1) = (0+1,0-1) = (1,-1)
\]

\(T_{1} \Rightarrow{} A = \begin{bmatrix}T(e_{1})&T(e_{2})\end{bmatrix} = \begin{bmatrix}1&1\\1&-1\end{bmatrix}\)

\(T_{2}(e_1) = T_{2}(1,0) = (2(1), -3(0)) = (2,0)\)
\(T_{2}(e_2) = T_{2}(0,1) = (2(0), -3(1)) = (0, -3)\)

\(T_{2}\Rightarrow{}B=\begin{bmatrix}T_{2}(e_1)&T_{2}(e_{2}\end{bmatrix}=\begin{bmatrix}2&0\\0&-3\end{bmatrix}\)

\(T_{1}\circ{}T_{2} \Rightarrow{} AB = \begin{bmatrix}1&1\\1&-1\end{bmatrix}\begin{bmatrix}2&0\\0&-3\end{bmatrix} = \begin{bmatrix}2&-3\\2&3\end{bmatrix}\)

\(T_{2}\circ{}T_{1} \Rightarrow{} BA = \begin{bmatrix}2&0\\0&-3\end{bmatrix} \begin{bmatrix}1&1\\1&-1\end{bmatrix} = \begin{bmatrix}2&2\\-3&3\end{bmatrix}\)

Transpose of each other
\section{Example}
\label{sec:org36f55d8}
\(T_{1}: \mathbb{R}^{2}\rightarrow{}\mathbb{R}^{3}\) by \(T_{1}(x,y) = (-2,2y,y)\)

\(T_{1}(e_1)=T_{1}(1,0)=(-1,0,0), T_{1}(e_{2})=T_{1}(0,1) = (0,2,1)\Rightarrow{}T_{1}\Rightarrow{}A=\begin{bmatrix}T_{1}(e_1)&T_{1}(e_{2})\end{bmatrix}=\begin{bmatrix}-1&0\\0&2\\0&1\end{bmatrix}\)
\(T_{2}(e_{1}) = T_{2}(1,0,0) = (-1,0,0), T_{2}(e_{2}) = T_{2}(0,1,0) = (0,-1,1), T_{2}(e_{3})=T_{2}(0,0,1) = (0,0,1) \Rightarrow{} T_{2} \Rightarrow{} B = \begin{bmatrix}T_{2}(e_1)&T_{2}(e_2)&T_{3}(e_3)\end{bmatrix} = \begin{bmatrix}3&0&0\\0&-1&0\\1&1&1\end{bmatrix}\)
\section{Prove that the composition of two rotations in \(\mathbb{R}^{2}\) is commutative}
\label{sec:org132fcea}
\begin{itemize}
\item \(\alpha\) + \(\beta\) degrees
\end{itemize}
Recall, supposed we rotate the vector \(\alpha\) \& \(\beta\) ccw

\(\alpha = \begin{bmatrix}cos\alpha{}&-sin\alpha{}\\sin\alpha{}&cos\alpha{}\end{bmatrix}\)
\(\beta = \begin{bmatrix}cos\beta{}&-sin\beta{}\\sin\beta{}&cos\beta{}\end{bmatrix}\)

\(\alpha{} + \beta{} \Rightarrow{} \begin{bmatrix}cos(\alpha{}+\beta{})&-sin(\alpha{}+\beta{})\\sin(\alpha{}+\beta{})&cos(\alpha{}+\beta{})\end{bmatrix}\)

wts \(AB = BA\)

\$ = \begin{bmatrix}cos(\(\alpha\)+\(\beta\))\&-sin(\(\alpha\)+\(\beta\))$\backslash$\(\sin\)(\(\alpha\)+\(\beta\))\&cos(\(\alpha\)+\(\beta\))\end{bmatrix}\$
\section{Reflecting a Vector across the line y=x}
\label{sec:orgbcf7952}
\(\begin{bmatrix}0&1\\1&0\end{bmatrix}\)
Transpose if you switch the order of the mat mul
\section{If \(T_{A}: \mathbb{R}^{n} \rightarrow{} \mathbb{R}^{n}\) is a matrix operator whose standard matrix A is invertible, then we say that T\textsubscript{A} is invertible, and we define the inverse T\textsubscript{A} as}
\label{sec:org4974ce9}
\[
T_{A}^{-1} = T_{A^{-1}}
\]

\begin{enumerate}
\item Given the operator \(T: \mathbb{R}^{2} \rightarrow{} \mathbb{R}^{2}\) defined by
\end{enumerate}
\[
w_{1} = 2x_{1} + 5x_{2}
w_{2} = -x_{1} + 7x_{2}
\]

compute T\textsuperscript{-1}(w\textsubscript{1},w\textsubscript{2})

WTF A\textsuperscript{-1}. So we need to find A first.
T(1,0)=(2(1)+5(0),-1+7(0) = (2,-1)
T(0,1) = (5,7)
\(\begin{bmatrix}2&5\\-1&7\end{bmatrix} \Rightarrow{} \frac{1}{19}\begin{bmatrix}7&-5\\1&2\end{bmatrix}\)
\((\frac{7}{19}w_{1}-\frac{5}{19}w_{2}, \frac{1}{19}w_{1} + \frac{2}{19}w_{2})\)
\end{document}
