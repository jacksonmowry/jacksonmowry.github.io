% Created 2024-10-18 Fri 09:45
% Intended LaTeX compiler: pdflatex
\documentclass[11pt]{article}
\usepackage[utf8]{inputenc}
\usepackage[T1]{fontenc}
\usepackage{graphicx}
\usepackage{longtable}
\usepackage{wrapfig}
\usepackage{rotating}
\usepackage[normalem]{ulem}
\usepackage{amsmath}
\usepackage{amssymb}
\usepackage{capt-of}
\usepackage{hyperref}
\date{\today}
\title{Three Three}
\hypersetup{
 pdfauthor={},
 pdftitle={Three Three},
 pdfkeywords={},
 pdfsubject={},
 pdfcreator={Emacs 29.4 (Org mode 9.7.11)}, 
 pdflang={English}}
\begin{document}

\maketitle
\tableofcontents

\begin{itemize}
\item Determine whether 2 vectors are parallel, orthogonal or neither
\item Compute the projection of one vector onto another
\item Decompose one vector into a sum of orthogonal and parallel parts
\item Compute distances between points and lines or points and planes
\item Compute a unit vector orthogonal to two vectors in R\textsuperscript{3}
\item Compute a unit vector orthogonal to any number of vectors in R\textsuperscript{n}
\end{itemize}
\section{Definition}
\label{sec:orgc4aa64f}
Two nonzero vectors u and v in R\textsuperscript{n} are orthogonal (or perpendicular) if u \(\cdot\) v = 0
\begin{itemize}
\item \(cos\theta{} = \frac{u \cdot{} v}{||u||| |v||} = 0\)
\end{itemize}
The zero vector in R\textsuperscript{n} is orthogonal to every vector in R\textsuperscript{n}
A nonempty set of vectors is called an orthogonal set if all pairs of distinct vectors in the set are orhtogonal
An orthogonal set of unit vectors is called an orthogonal set
\begin{itemize}
\item \({u_{1}, u_{2}, ..., u_{n}}\) is an orthogonal set
\begin{itemize}
\item Then u\textsubscript{i} \(\cdot\) u\textsubscript{j} = 0 \(\forall\) i,j \(\in\) \{1, n\}
\end{itemize}
\end{itemize}
R\textsuperscript{3} = i, j, k \(\Rightarrow\) \{i, j, k\}
\subsection{Example}
\label{sec:org8e562e0}
The standard unit vectors in any space
\begin{itemize}
\item \(R^{n} = \{e_{1}, e_{2}, ..., e_{n}\}\) = orthogonal set
\item \(R^{2} = \{(1,0), (0,1)\}\)
\item \(R^{3} = \{(1,0,0), (0,1,0), (0,0,1)\}\)
\end{itemize}
\subsection{Make it Orthogonal}
\label{sec:orga23a464}
\begin{itemize}
\item \((1,2,3) \cdot (-3,0,1) = 0\)
\item \((1,2,3) \cdot (1,-5,3) = 0\)
\item \((-3,0,1) \cdot (1,-5,3) = 0\)
\end{itemize}
It is already orthogonal
\section{Point-Normal Equations of Lines and Planes}
\label{sec:org6ea0865}
One application of orthogonality is to define a line or place by using a nonzero vector n, called a normal, that is orthogonal to the line or plane

For example \(n \cdot P_{0}P_{1} = 0\), where \(P_{0}P_{1} = (x-x_{0}, y-y_{0})\) or \(P_{0}P_{1} = (x-x_{0}, y-y_{0}, z-z_{0})\)

Find the equation of the plane the goes through the origin and has normal (10, 3, 13)
\begin{itemize}
\item Equation of a plane through a point \((x_{0}, y_{0}, z_{0})\) and has normal \((a,b,c)\)
\item \(a(x-x_{0}) + b(y-y_{0}) + c(z-z_{0}) = 0\)
\item \(10x + 3y + 13z = 0\)
\end{itemize}

Find the equation of the plane that goes through the point (1,2,3) and has normal (10,3,13)
\begin{itemize}
\item \(10(x-1) + 3(y-2) + 13(z-3) = 0\)
\end{itemize}
\section{Orthogonal Projections}
\label{sec:org1fce40f}
If u and a are vectors in R\textsuperscript{n} and if a \(\neq\) 0, then u = w\textsubscript{1} + w\textsubscript{2}, hwere w\textsubscript{1} is a scalar multiple of a and w\textsubscript{2} is orthogonal to a

\begin{itemize}
\item w\textsubscript{1} is the vector component of u along a
\begin{itemize}
\item \(w_{1} = proj_{u}(u) = \frac{u\cdot{}a}{a\cdot{}a}a\)
\end{itemize}
\item and w\textsubscript{2} is the vector component of u orthogonal to a
\begin{itemize}
\item \(w_{2} = u - proj_{a}(u) = u - \frac{u\cdot{}a}{a\cdot{}a}a\)
\end{itemize}
\end{itemize}
\subsection{Example}
\label{sec:org89458b6}
u = (1,9,0,5), a = (3,-2,4,1)
\begin{itemize}
\item \(w_{1} = proj_{a}u = \frac{u\cdot{}a}{a\cdot{}a}a = \frac{1(3)+9(-2)+0(-4)+5(1)}{3^{2}+-2^{2}+4^{2}-1^{2}} \times{} (3,-2,4,1) = \frac{-10}{30}(3,-2,4,1) = (-1, \frac{2}{3}, \frac{-4}{3}, \frac{-1}{3})\)
\item \(w_{2} = u - proj_{a}u = (1,9,0,5) - (-1, \frac{2}{3}, \frac{-4}{3}, \frac{-1}{3}) = (2, \frac{25}{3}, \frac{4}{3}, \frac{16}{3})\)
\end{itemize}
\subsection{Another one}
\label{sec:org9aed090}
u=(2,1,-4,6) and a=(1,-2,2,1)
\begin{itemize}
\item \(w_{1} = \frac{2(1)+1(-2)+(-4)(2)+6(1)}{1^{2}+(-2)^{2}+2^{2}+1^{2}} \times{} (1,-2,2,1) = \frac{-2}{10}(1,-2,2,1) = (\frac{-1}{5},\frac{2}{5},\frac{-2}{5},\frac{-1}{5})\)
\item \(w_{2} = (2,1,-4,6) - (\frac{-1}{5},\frac{2}{5},\frac{-2}{5},\frac{-1}{5}) = (\frac{11}{5}, \frac{3}{5}, \frac{-18}{5}, \frac{31}{5})\)
\end{itemize}
\subsection{Checks}
\label{sec:org357124f}
\subsubsection{Problem 1}
\label{sec:org5aadb64}
w\textsubscript{2} \(\cdot\) a = ?
This should be orthogonal component
\subsubsection{Problem 2}
\label{sec:org6a78856}
w\textsubscript{2} \(\cdot\) a = ?
This should be orthogonal component
\subsection{Finding the Norm}
\label{sec:org13c5902}
\begin{itemize}
\item \(\sqrt{1^2 + \frac{2}{3}^{2} + }\)
\end{itemize}
\section{Pythag}
\label{sec:org76a3e9b}
in R\textsuperscript{n}: If u and v are orthogonal vectors in R\textsuperscript{n} with the Euclidean inner product, then ||u+v||\textsuperscript{2} = ||u||\textsuperscript{2} + ||v||\textsuperscript{2}

We know that ||a||\textsuperscript{2} = a \(\cdot\) a

\begin{itemize}
\item ||u + v||\textsuperscript{2} = (v + v) \(\cdot\) (u + u) = u\(\cdot\)u + u\(\cdot\)v + v\(\cdot\)v + v\(\cdot\)u
\end{itemize}
\subsection{find the distance between a point (x\textsubscript{0},y\textsubscript{0}) and a line ax+by+c = 0 in R\textsuperscript{2}}
\label{sec:org51fba3b}
\begin{itemize}
\item \(D = \frac{|ax_{0}+by_{0}+c|}{\sqrt{a^2 + b^2}}\)

\item \(D = \frac{|5(2)-11-9|}{\sqrt{5^2 + (-1)^{2}}} = \frac{10}{\sqrt{26}}\)
\end{itemize}
\subsection{find a distance between a point and a plane in R\textsuperscript{3}}
\label{sec:org64239d3}
\begin{itemize}
\item \(D = \frac{|ax_{0}+by_{0}+cz_{0}+d|}{\sqrt{a^{2} + b^{2} + c^{2}}}\)
\item \(D = \frac{|2(3)+3(5)-10+4|}{\sqrt{2^{2} + 3^{2} + (-1}^{2})}} = \frac{15}{\sqrt{14}}\)
\end{itemize}
\end{document}
