% Created 2024-09-16 Mon 08:37
% Intended LaTeX compiler: pdflatex
\documentclass[11pt]{article}
\usepackage[utf8]{inputenc}
\usepackage[T1]{fontenc}
\usepackage{graphicx}
\usepackage{longtable}
\usepackage{wrapfig}
\usepackage{rotating}
\usepackage[normalem]{ulem}
\usepackage{amsmath}
\usepackage{amssymb}
\usepackage{capt-of}
\usepackage{hyperref}
\date{\today}
\title{Exam Review 1}
\hypersetup{
 pdfauthor={},
 pdftitle={Exam Review 1},
 pdfkeywords={},
 pdfsubject={},
 pdfcreator={Emacs 29.4 (Org mode 9.7.11)}, 
 pdflang={English}}
\begin{document}

\maketitle
\tableofcontents

\section{1.1}
\label{sec:org8238c1f}
\begin{itemize}
\item Gauss-Jordan Elimination
\begin{itemize}
\item Add a multiple of one row to another
\item Swap 2 rows
\item Multiply a row by a constant
\end{itemize}
\end{itemize}
\section{1.2}
\label{sec:orgfe1a79a}
\begin{itemize}
\item Solve system to get solution with parameters
\begin{itemize}
\item Use Gauss-Jordan to get rref
\end{itemize}
\item How do we see if the system has n solutions
\begin{enumerate}
\item If after doing gauss-jordan elimination, the last row is all 0s, and the rhs is non-zero
\begin{itemize}
\item No solutions
\end{itemize}
\item All zeros but one of the lhs vars is not
\begin{itemize}
\item X\textsubscript{n} = 0
\end{itemize}
\item \# of nonzero rows < \# variables
\begin{itemize}
\item Infinite solutions
\item \# of variables - number of non-zero rows
\end{itemize}
\end{enumerate}
\end{itemize}
\section{1.3}
\label{sec:org0332a6a}
\begin{itemize}
\item Write a matrix product as a linear combination of columns
\begin{itemize}
\item \$ Ax\textsuperscript{\(\rightarrow\)} => column\textsubscript{1}x\textsubscript{1} + column\{2\}x\textsubscript{2} + \ldots{} + column\{n\}x\textsubscript{n}
\end{itemize}
\item Do matrix multiplication
\begin{itemize}
\item A\textsubscript{m x \textbf{n}}B\textsubscript{\textbf{n} x o} => (AB)\textsubscript{m x l}
\item Row of first matrix, times column of second matrix
\end{itemize}
\item Compute the trace
\end{itemize}
\section{1.4}
\label{sec:orgaa1ca78}
\begin{itemize}
\item Find inverse of A and solve Ax\textsuperscript{\(\rightarrow\)} = b\textsuperscript{\(\rightarrow\)}
\item A\textsubscript{3 x 3} => [A | I] -> gauss-jordan -> [I | A\textsuperscript{-1}], after having A\textsuperscript{-1}, Ax\textsuperscript{\(\rightarrow\)} = b\textsuperscript{\(\rightarrow\)}
\item Apply inverses to isolate variables
\end{itemize}
\section{1.5}
\label{sec:org4f0d2d8}
\begin{itemize}
\item Find E so that EA = B, where A\&B are given
\begin{itemize}
\item Realize the change in A to get B, a single row operation
\end{itemize}
\end{itemize}
\section{1.6}
\label{sec:org116edf7}
\begin{itemize}
\item Find b for Ax\textsuperscript{\(\rightarrow\)} = b\textsuperscript{\(\rightarrow\)} to be consistent
\item Given Ax\textsuperscript{\(\rightarrow\)} = b\textsuperscript{\(\rightarrow\)}
\begin{itemize}
\item {[}A | b\textsuperscript{\(\rightarrow\)}] => gauss-jordan elimination
\item \(\begin{bmatrix}*&(any)&(any)\\0&*&(any)\\0&0&*\end{bmatrix}\)
\item Analize the last row to
\end{itemize}
\item m = 0, n is an expression of b\textsubscript{1},b\textsubscript{2}, and b\textsubscript{3}
\item then set n = 0, solve for b\textsubscript{1},b\textsubscript{2}, and b\textsubscript{3}
\item Use parameters t, s (if needed) to write the solution
\begin{itemize}
\item \$b\textsuperscript{\(\rightarrow\)} = t[] + s[]
\end{itemize}
\end{itemize}
\section{1.7}
\label{sec:org3b58fac}
\begin{itemize}
\item Just memorize things
\begin{itemize}
\item Diagonal matrix
\item Triangular matrix
\item Skew matrix
\item Symmetric matrix
\end{itemize}
\item A\textsubscript{n x n}, B\textsubscript{n x n}
\begin{itemize}
\item (AB)\textsuperscript{T} = B\textsuperscript{T}A\textsuperscript{T}
\item (AB)\textsuperscript{-1} = B\textsuperscript{-1}A\textsuperscript{-1}
\item (A\textsuperscript{-1})\textsuperscript{T} = (A\textsuperscript{T})\textsuperscript{-1}
\end{itemize}
\end{itemize}
\end{document}
