% Created 2024-09-15 Sun 16:03
% Intended LaTeX compiler: pdflatex
\documentclass[11pt]{article}
\usepackage[utf8]{inputenc}
\usepackage[T1]{fontenc}
\usepackage{graphicx}
\usepackage{longtable}
\usepackage{wrapfig}
\usepackage{rotating}
\usepackage[normalem]{ulem}
\usepackage{amsmath}
\usepackage{amssymb}
\usepackage{capt-of}
\usepackage{hyperref}
\date{\today}
\title{One Six}
\hypersetup{
 pdfauthor={},
 pdftitle={One Six},
 pdfkeywords={},
 pdfsubject={},
 pdfcreator={Emacs 29.4 (Org mode 9.7.11)}, 
 pdflang={English}}
\begin{document}

\maketitle
\tableofcontents

\section{Objectives}
\label{sec:org8fb0dab}
\begin{itemize}
\item Expanding the Equivalency Theorem
\item Given a singular matrix A, find all matricies b such that Ax = b is consistent
\end{itemize}
\section{Theorem}
\label{sec:org70926f2}
If A is an n x n matrix, then the following statements are equivalent
\begin{enumerate}
\item A is invertible
\item Ax = 0 has only the trivial solution
\item The reduced row-echelon form of A is I\textsubscript{n}
\item A is expressible as a product of elementary matricies
\item Ax = b is consistent for each n x 1 matrix b
\item Ax = b has exactly one solution for each n x 1 matrix b
\end{enumerate}
\section{Theorem}
\label{sec:org8d0947f}
For A and B square matricies of the same size, if AB is invertible, then A and B must also be invertible
\subsection{Proof}
\label{sec:orga280bed}
\section{Problem}
\label{sec:org22be1ff}
If A is either not square or not invertible, find all b such that Ax = b is consistent

If a is invertible then it has 1 solution
If a is non-invertible then it either has infinite solutions or no solutions
\subsection{Examples}
\label{sec:orgda218d6}
\begin{enumerate}
\item Find all b such that \(x+3y=b_{1}\) \(2x+6y=b_{2}\) has a solution
\end{enumerate}
\(\begin{bmatrix}1&3\\2&6\end{bmatrix}\)
r\textsubscript{2} - 2r\textsubscript{1} \(\rightarrow\) r\textsubscript{2}
\(\begin{bmatrix}1&3\\0&0\end{bmatrix}\)

The last row tells us that \(0x_1 + 0x_2 = b_2 - 2b_1\), which can only be true if \(b_2 = 2b_1\)

So, if \(b_2 - 2b_1 \neq{} 0\), then we have no solutions. Thus, we would have solutions when \(b_2 - 2b_1 = 0\)

How to write a single column \(b^{\rightarrow{}} = \begin{bmatrix}b_1\\2b_1\end{bmatrix}\)

\begin{enumerate}
\item Find all b such that \(x+3y+2z=b_{1}\) \(2x+7y+4z=b_{2}\) \(4x+5y+8z=b_{3}\)
\end{enumerate}
\(\begin{bmatrix}1&3&2\\2&7&4\\4&5&8\end{bmatrix}\)
\(\begin{bmatrix}1&3&2\\2&7&4\\4&5&8\end{bmatrix}\)

\(\left[\begin{array}{ccc|c}1&3&2&b_1\\0&1&0&b_{2}-2b_{1}\\0&0&0&b_{3}-4b_{1}+7(b_{2}-2b_{1}\end{array}\right]\)

No solution unless \(b_3-4b_1+7(b_2-2b_1) = 0\)

Let \(b_1 = t, b_2 = s\)

\(b_3 - 18t + 7s = 0\)
\(b_3 = 18t - 7s\)

Thus, \(b = \begin{bmatrix}b_1\\b_2\\b_3\end{bmatrix} = \begin{bmatrix}t\\s\\18t-7s\end{bmatrix} = \begin{bmatrix}t\\0\\18t\end{bmatrix} + \begin{bmatrix}0\\s\\-7s\end{bmatrix} = t\begin{bmatrix}1\\0\\18\end{bmatrix} + s\begin{bmatrix}0\\1\\-7\end{bmatrix}\)

When you add 2 vectors you add component wise, in the above column vector we have 2 variables, so we write in 2 separate column vectors. First entry we have t, so pull it out
\end{document}
