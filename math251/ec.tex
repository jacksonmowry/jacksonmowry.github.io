% Created 2024-10-27 Sun 11:46
% Intended LaTeX compiler: pdflatex
\documentclass[11pt]{article}
\usepackage[utf8]{inputenc}
\usepackage[T1]{fontenc}
\usepackage{graphicx}
\usepackage{longtable}
\usepackage{wrapfig}
\usepackage{rotating}
\usepackage[normalem]{ulem}
\usepackage{amsmath}
\usepackage{amssymb}
\usepackage{capt-of}
\usepackage{hyperref}
\author{Jackson Mowry}
\date{Sun Oct 27 10:38:59 2024}
\title{Ec Vector Spaces}
\hypersetup{
 pdfauthor={Jackson Mowry},
 pdftitle={Ec Vector Spaces},
 pdfkeywords={},
 pdfsubject={},
 pdfcreator={Emacs 29.4 (Org mode 9.7.11)}, 
 pdflang={English}}
\begin{document}

\maketitle
\section{6}
\label{sec:orgc00514a}
The set V = \{(x,y,z) | x,y,z are integers\}, with regular component addition and scalar multiplication as the operations\\
\begin{enumerate}
\item ku \(\in\) V (Axiom 6)\\
\begin{itemize}
\item Scalar multiples do not have to be integers \(\therefore\) ku \(\notin\) V\\
\item \(u = (1,1,1), k = 4.5 \therefore ku = (4.5,4.5,4.5) \notin{} V\)\\
\end{itemize}
\end{enumerate}
\section{7}
\label{sec:org34c2579}
The set V of all continuous functions that are differentiable on (-\(\inf\)(), \(\inf\))\\
\begin{enumerate}
\item u + v \(\in\) V\\
\begin{itemize}
\item Both u and v are continuous and differentiable, u+v must also be continuous and differentiable (as the sum of continuous and differentiable functions respectively)\\
\item u + v \(\in\) V\\
\end{itemize}
\item u + v = v + u\\
\begin{itemize}
\item Function addition is known to be commutative\\
\item Therefore u + v = v + u\\
\end{itemize}
\item u + (v + w) = (u + v) + w\\
\begin{itemize}
\item Function addition is known to be associative\\
\item Therefore u + (v + w) = (u + v) + w\\
\end{itemize}
\item There exists a zero 0 \overrightarrow{0} such that 0 + u = u + 0 = u\\
\begin{itemize}
\item The zero function f(x) = 0 for all x is continuous and differentiable, thus the zero vector is in V\\
\end{itemize}
\item For each u in V, there exists -u such that u + -u = 0\\
\begin{itemize}
\item If a function f is continuous and differentiable, then -f must also be continuous and differentiable, thus -f \(\in\) V\\
\end{itemize}
\item ku \(\in\) V\\
\begin{itemize}
\item If u is continuous and differentiable, then ku is continuous (since scalar multiples of continuous functions are continuous) and differentiable (since scalar multiples of differentiable functions are differentiable)\\
\item \(\therefore\) ku \(\in\) V\\
\end{itemize}
\item k(u + v) = ku + kv\\
\begin{itemize}
\item This follows the basic properties of scalar multiplication and function addition\\
\item \(\therefore\) this property holds for our vector space\\
\end{itemize}
\item (k+m)u = ku + mu\\
\begin{itemize}
\item This is the same property as scalar multiplication\\
\item \((k+m)u\)\\
\item \(uk + um\)\\
\item \(ku + mu\)\\
\end{itemize}
\item (km)u = k(mu)\\
\begin{itemize}
\item This follows the same properties as scalar multiplication as proven above\\
\end{itemize}
\item 1 \texttimes{} u = u\\
\begin{itemize}
\item We can simply substitute the value 0 in for '1' in the original axiom\\
\begin{itemize}
\item Or the zero function\\
\end{itemize}
\item This is treated as an addition of the identity\\
\item \(f(x) + 0 = f(x)\)\\
\end{itemize}
\end{enumerate}
\end{document}
