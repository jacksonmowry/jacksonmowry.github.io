% Created 2025-11-05 Wed 12:21
% Intended LaTeX compiler: pdflatex
\documentclass[11pt]{article}
\usepackage[utf8]{inputenc}
\usepackage[T1]{fontenc}
\usepackage{graphicx}
\usepackage{longtable}
\usepackage{wrapfig}
\usepackage{rotating}
\usepackage[normalem]{ulem}
\usepackage{amsmath}
\usepackage{amssymb}
\usepackage{capt-of}
\usepackage{hyperref}
\author{UTVerse}
\date{\today}
\title{Review 3}
\hypersetup{
 pdfauthor={UTVerse},
 pdftitle={Review 3},
 pdfkeywords={},
 pdfsubject={},
 pdfcreator={Emacs 30.2 (Org mode 9.7.31)}, 
 pdflang={English}}
\begin{document}

\maketitle
\section*{Summary}
\label{sec:org9f34827}

The document discusses the RISP neuroprocessor, an open-source tool designed to support research and development in neuromorphic computing. Neuromorphic computing is a technology inspired by the way the human brain processes information, using artificial neurons and synapses to create efficient, low-power systems. Unlike traditional computers that rely on large memory and high power, neuromorphic processors use spiking neural networks (SNNs), which are more energy-efficient and computationally powerful.\\
Key Points:\\
\subsection*{What is Neuromorphic Computing?}
\label{sec:org3a151fe}
It mimics the brain’s structure and function, using neurons and synapses to process information.\\
Spiking Neural Networks (SNNs) add a temporal aspect to computations, making them distinct from traditional artificial neural networks.\\
\subsection*{The RISP Neuroprocessor:}
\label{sec:orgcb9637b}
A simplified neuromorphic processor that uses “integrate-and-fire” neurons and synapses with discrete delays.\\
It enables researchers to simulate and implement neuromorphic networks on hardware like Field-Programmable Gate Arrays (FPGAs).\\
Two repositories are available:\\
Framework-open: For simulation and network manipulation.\\
FPGA repository: For implementing RISP networks on FPGAs.\\
\subsection*{Applications:}
\label{sec:org070fc0c}
Examples include solving problems like balancing a cart-pole system, computing mathematical functions like sine, recognizing patterns (bars-and-stripes), and clustering data using DBSCAN algorithms.\\
These applications demonstrate how neuromorphic computing can handle complex tasks efficiently.\\
\subsection*{Advantages of RISP:}
\label{sec:org8aebb37}
Open-source tools make it accessible for researchers without requiring expensive hardware or proprietary solutions.\\
The approach focuses on embedding specific networks directly into FPGA hardware, optimizing resource usage compared to general-purpose neuroprocessors.\\
\subsection*{Challenges:}
\label{sec:org0b86b56}
Communication between the FPGA and host computer is currently a bottleneck due to limitations in UART communication protocols.\\
The authors propose solutions like buffering, aggregation, and advanced communication protocols (e.g., AXI DMA) to improve performance.\\
\subsection*{Future Goals:}
\label{sec:org810bc9f}
Enhance communication efficiency for real-time applications like event-based cameras.\\
Expand support for additional FPGA platforms and communication protocols.\\
Encourage contributions from the open-source community to improve the implementation further.\\

In summary, this paper introduces the RISP neuroprocessor as a practical tool for exploring neuromorphic computing in embedded systems. By leveraging open-source software and hardware implementations, researchers can create energy-efficient systems capable of handling complex tasks while addressing current challenges in communication efficiency.\\
\section*{My Thoughts}
\label{sec:org406f5ce}
\begin{itemize}
\item As compared to previous summaries I thought UTVerse did a really good job this week. It hit on all of the key points leaving very little left up to the reader to go discover independently.\\
\item If I had to change one thing I would include hard numbers on the communication bottle neck.\\
\item We also need to include the title and authors into the summary.\\
\end{itemize}
\end{document}
