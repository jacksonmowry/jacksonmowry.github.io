% Created 2024-10-13 Sun 14:20
% Intended LaTeX compiler: pdflatex
\documentclass[11pt]{article}
\usepackage[utf8]{inputenc}
\usepackage[T1]{fontenc}
\usepackage{graphicx}
\usepackage{longtable}
\usepackage{wrapfig}
\usepackage{rotating}
\usepackage[normalem]{ulem}
\usepackage{amsmath}
\usepackage{amssymb}
\usepackage{capt-of}
\usepackage{hyperref}
\date{\today}
\title{CS312 Notes}
\hypersetup{
 pdfauthor={},
 pdftitle={CS312 Notes},
 pdfkeywords={},
 pdfsubject={},
 pdfcreator={Emacs 29.3 (Org mode 9.7)}, 
 pdflang={English}}
\begin{document}

\maketitle
\tableofcontents

\section{Pushdown Automata}
\label{sec:orgc551a51}
A pushdown automata (or PDA) is similar to an NFA but it has a stack. The stack provides additional memory beyond finite memory available in control; it allows the PDA to recognize some nonregular languages.\\

2 options to prove htat a language is context-free\\
\begin{itemize}
\item Construct a CFG that generates it\\
\item Construct a PDA that recognizes it\\
\end{itemize}

Some CFLs are more easily descibed in terms of their generators, whereas others are more easily described in terms of their recognizers. Let's draw a schematic representation of the difference between an NFA and a PDA.\\
\subsection{Terminology}
\label{sec:orgeb4bfe1}
\begin{itemize}
\item Writing a symbol on the stack is called pushing the symbol\\
\item Removing a symbol from the stack is called popping the symbol\\
\item All access to the stack may be done only at the top (LIFO storage device)\\
\end{itemize}

The primary benefit of that stack is that it can hold an \textbf{unlimited} amount of data; a PDA can recognize \{0\textsuperscript{n}1\textsuperscript{n} | n \(\ge\) 0\} because it can use the stack to remember that number of 0s it has seen (read)\\
\subsection{Informal Algorithm for \{0\textsuperscript{n}1\textsuperscript{n} | n \(\ge\) 0\}}
\label{sec:org4072a82}
\begin{enumerate}
\item Read symbols from the input, push a 0 for each 0 you see\\
\item As soon as a 1 is read, pop a 0 off the stack (for each 1 read).\\
\item If input finishes when the stack become empty, accept; if stack becomes empty while there is still input or input finishes while the stack is not empty, reject.\\
\end{enumerate}

A PDA may be nondeterminisitic. Languages as the one above do not require nondeterminism. However, the language \{ww\textsuperscript{R} | w \(\in\) \{0,1\}\textsuperscript{*}\} would require nondeterminism. Why?\\
\subsection{Formalization}
\label{sec:orgd64be42}
\begin{itemize}
\item A PDA may different alphabets for input (\(\Sigma\)) and stack (\(\Gamma\))\\
\item Nondeterminism allows for the PDA to make transitions on empty input. Define \(\Sigma\)\textsubscript{\(\epsilon\)} = \(\Sigma\) \union \{\(\epsilon\)\} and \(\Gamma\)\textsubscript{\(\epsilon\)} = \(\Gamma\) \union{} \{\(\epsilon\)\}\\
\item The domain of the PDA transition function is Q x \(\Sigma\)\textsubscript{\(\epsilon\)} x \(\Gamma\)\textsubscript{\(\epsilon\)}, where Q is the set of states\\
\item The range of the PDA transition funciton is P(Q x \(\Gamma\)\textsubscript{\(\epsilon\)}).\\
\end{itemize}
\[
\delta{}: Q x \Sigma{}_{\epsilon{}} x \Gamma{}_{\epsilon} \rightarrow{} P(Q x \Gamma{}_{\epsilon})
\]\\
\subsection{Formal Def}
\label{sec:orgcf5d17a}
A PDA is a 6-tuple (Q, \(\Sigma\), \(\Gamma\), \(\delta\), q\textsubscript{0}, F), where Q, \(\Sigma\), \(\Gamma\) are finite sets of states\\
\subsection{PDA Computation}
\label{sec:orgd9d14c8}
A PDA M = (Q,\(\Sigma\),\(\Gamma\),\(\delta\),q\textsubscript{0},F) computes as follows\ldots{}\\
M inputs w = w\textsubscript{1}w\textsubscript{2}\ldots{}w\textsubscript{m}, where each w\textsubscript{i} \(\in\) \(\Sigma\)\textsubscript{\(\epsilon\)}\\
\begin{enumerate}
\item r\textsubscript{0} = q\textsubscript{0},s\textsubscript{0} = \(\epsilon\); begin with state state and empty stack\\
\item (r\textsubscript{i+1},b) \(\in\) \(\delta\)(r\textsubscript{i},w\textsubscript{i+1},a), i = 0,1,\ldots{}m-1, where s\textsubscript{i} = at and s\textsubscript{i+1} = bt for some a,b \(\in\) \(\Gamma\) and t \(\in\) \(\Gamma\)\textsuperscript{*}\\
\item r\textsubscript{m} \(\in\) F; accept state encountered at end of input\\
\end{enumerate}
\subsection{Stack Notation}
\label{sec:orgb559536}
a,b \(\rightarrow\) or simply abc\\
\begin{itemize}
\item a = input\\
\item b = what are you popping\\
\item c = push onto stack\\
\end{itemize}
A is read from the input, b is poppped from the stack, and c is pushed onto the stack\\
\(\epsilon\)-cases\\
\begin{itemize}
\item If a = \(\epsilon\), machine can transition without reading any input\\
\item if b = \(\epsilon\), machine can transition without popping any symbol from the stack\\
\item if c = \(\epsilon\), machine can transition without writing any symbol onto the stack\\
\end{itemize}

read pop push\\
\subsection{Empty Stack}
\label{sec:org3f3e88d}
The PDA does not consider the testing of an empty stack. We can achieve this by initially placing a special char (say \() on the stack. When the PDA encounters that char (\)) again (on the stack), it knows the stack is effectively empty.\\

Both CFGs and PDAs specify context-free languages; we can always convert a CFG into a PDA that recognizes the language of the CFG\\
\subsection{Theorem}
\label{sec:orge819a68}
CFG - specifies a program language\\
PDA - specifies/implements the compiler\\

A language is context-free \iff{} some PDA recognizes it\\
\subsection{Difficulties}
\label{sec:org8a5cbdb}
How do we decide which substitutions to make for a derivation? (PDA P nondeterminism can help)\\
\begin{itemize}
\item At each step of the derivation one of the rules for a particular variable is selected non-deterministically\\
\item P has to start by writing the start variable on the stack and then continue working the string w\\
\item If while consuming the string w, P arrives at a string of terminals that equals w\\
\end{itemize}
\subsection{Informal Description}
\label{sec:orgdfe8183}
Place marker symbol \$ and tart varaible on the stack\\
Repeat:\\
\begin{itemize}
\item If TOS is a variable symbol A, non-deterministically select a rule r such that lhs(r) = A and substitute A by the string rhs(s)\\
\item If TOS is a terminal symbol, a, read the next input symbol and compare it with a; if they match, pop the stack; if they do not match, reject this branch of nondeterminism\\
\item If TOS is a \$ and all the text has been read, accept; otherwise reject\\
\end{itemize}
\subsection{Generic State Diagram}
\label{sec:org84812f3}
\begin{enumerate}
\item TOS = variable: set \(\delta\)(q\textsubscript{loop}, \(\epsilon\), A) = \{(q\textsubscript{loop}, w) | A \(\rightarrow\) w \(\in\) R\}, where R is the set of CFG rules\\
\item TOS = terminal: set \(\delta\)(q\textsubscript{loop}, a, a\}) = \{(q\textsubscript{loop}, \(\epsilon\)\})\\
\item TOS = \$: \(\delta\)(q\textsubscript{loop}, \(\epsilon\), \$) = \{(q\textsubscript{accept}, \(\epsilon\))\}\\
\end{enumerate}
\end{document}
