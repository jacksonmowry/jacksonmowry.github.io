% Created 2025-01-31 Fri 09:45
% Intended LaTeX compiler: pdflatex
\documentclass[11pt]{article}
\usepackage[utf8]{inputenc}
\usepackage[T1]{fontenc}
\usepackage{graphicx}
\usepackage{longtable}
\usepackage{wrapfig}
\usepackage{rotating}
\usepackage[normalem]{ulem}
\usepackage{amsmath}
\usepackage{amssymb}
\usepackage{capt-of}
\usepackage{hyperref}
\date{\today}
\title{Operating Systems}
\hypersetup{
 pdfauthor={},
 pdftitle={Operating Systems},
 pdfkeywords={},
 pdfsubject={},
 pdfcreator={Emacs 29.4 (Org mode 9.7.11)}, 
 pdflang={English}}
\begin{document}

\maketitle
\tableofcontents

\section{Chapter 2}
\label{sec:org5991405}
\begin{itemize}
\item We have abstractions that allow a single program to behave as the only process on the machine
\item Mechanisms are how is the code implemented
\item Policies are how we decide when to use the mechanisms
\item Primary way we expose resources is through virtualization
\begin{itemize}
\item CPU, RAM, disk
\item Make virtual forms for the program to use
\end{itemize}
\end{itemize}
\subsection{Evolution}
\label{sec:org3f2ef08}
\begin{itemize}
\item Early OS was just a lib, only one process at a time
\item Then we gain protections, raw data storage is dangerous
\item Then we gained multiprogramming. Running more than one job at a time
\item As we talk about various OS pieces, we will show the evolution into the modern versions .
\end{itemize}
\section{Chapter 8}
\label{sec:org86c543d}
\subsection{MLFQ: Multilevel Feedback Queue}
\label{sec:orgee9d062}
\begin{itemize}
\item Multiple queues
\item each is assigned a priority level
\item When we make a decision we go to highest priority
\begin{itemize}
\item For jobs within the same queue we use round-robin
\end{itemize}
\end{itemize}
\subsubsection{Rules}
\label{sec:org478d466}
\begin{enumerate}
\item If priority(A) > priority(B) => A
\item If priority(A) == priority(B) => Round-robin
\item When a job enters it is at highest priority
\item If it uses its entire timeslice its priority is reduced, no matter how many times it gives up the CPU
\item If it gives up the CPU before its timeslice is up it stays
\item After some time point S, move all jobs in the system to the topmost queue
\end{enumerate}
\subsection{How to select Priority?}
\label{sec:org5761c3b}
\begin{itemize}
\item Job that often yields (getting keyboard input) should be prioritized for quicker response time
\item If a job keeps making syscalls, we should priortize it as it is probably an interactive process
\item If it uses the cpu a lot we should prioritize turn-around time
\end{itemize}

But job priority cannot be entirely static, it must be able to change over time.
\subsection{How to change Priority?}
\label{sec:org376f7d4}
\begin{itemize}
\item Give each job an allotment, this is the amount of time it can exist in a queue without reducing its priority
\item When a job begins it is placed at the highest priority
\item If it uses its entire timeslice its priority is reduced
\item If a job gives up the CPU before running through its allotment it can stay at the same priority
\end{itemize}

Highly interactive jobs will always stay at the highest priority because they never use up an entire timeslice without making a syscall

A program can also game the scheduler by always issuing IO right before the time slice is up

A program can also change its behavior from CPU intensive to IO intensive
\subsection{Priority Boost}
\label{sec:orgef2b7bf}
Process could be getting starved, but if we boost everything, they will all then get a fair chance to run again
\subsection{Better Accounting}
\label{sec:orgc462c34}
\begin{itemize}
\item To prevent gaming, change rule 4
\item If you use up your 10ms timeslice (even across schedules) then you move down
\end{itemize}
\section{Chapter 9}
\label{sec:org783703c}
Proportional Share Scheduling
\subsection{A New Metric}
\label{sec:org32b8f75}
\begin{itemize}
\item Tickets represent the share of a resource that an entity should recieve
\item The percent of tickets it hold represent the \% they get
\end{itemize}
\subsubsection{Lottery}
\label{sec:org876fbbe}
\begin{itemize}
\item Draw a random ticket, schedule the process that holds that ticket
\item More tickets you have the more likely you are to be scheduled
\end{itemize}
\subsubsection{Ticket Currency}
\label{sec:org4f7d4cf}
\begin{itemize}
\item Allows a user to create their own currency
\begin{itemize}
\item Global currency and then whatever currency each user creates
\end{itemize}

\item Example both A and B are given 100 global tickets
\begin{itemize}
\item A runs A1 and A2
\begin{itemize}
\item A1/A2 both get 500 A bucks
\end{itemize}
\item B runs B1
\begin{itemize}
\item B1 gets 10 B bucks
\end{itemize}

\item A1 and A2 are getting 50 global tickets each
\item B1 gets 100 global tickets
\end{itemize}
\end{itemize}
\subsubsection{Ticket Transfer}
\label{sec:orga4ade9d}
\begin{itemize}
\item allows a process to lend tickets to another process
\item Client/server running on same machine
\begin{itemize}
\item Client sends request, wait on server
\item So it also send some of its tickets over to the server as well
\item The server will send them back when done
\end{itemize}
\end{itemize}
\subsubsection{Ticket Inflation}
\label{sec:orgad68ec6}
\begin{itemize}
\item Can just grow its own ticket amount, only really makes sense in a co-op environment
\end{itemize}
\subsubsection{Impl}
\label{sec:orge3532a6}
\begin{itemize}
\item Good RNG
\item List to hold processes
\item Total \# of tickets

\item Generate number N
\item Traverse list, add up ticket values
\item Winner once the total is greater than N

\item Linear time
\end{itemize}
\subsubsection{Fairness}
\label{sec:org3ceb002}
\begin{itemize}
\item 2 jobs of the same length how fair is the scheduler?
\item Randomness affects short jobs
\item Fairness = job\textsubscript{finish}\textsubscript{1} / job\textsubscript{finish}\textsubscript{2}
\item Want fairness to be 1
\end{itemize}
\subsection{Stride Scheduling}
\label{sec:org5344723}
\begin{itemize}
\item Randomness isn't fair
\item Deterministic fair scheduler
\item Stride = (some large number) / tickets
\begin{itemize}
\item Inverse in proportion to the number of tickets it has
\end{itemize}
\item Each process has a running pass value starting at 0
\item When a process is scheduled, increment its pass by stride
\item Always schedule the process with the lowest pass breaking ties arbitrarily

\item Why do we even do this?
\begin{itemize}
\item We now have global state, lottery doesnt
\item What happens if we add a new process? What should its pass be? Do we start at 0, or start at average
\end{itemize}
\end{itemize}
\subsection{Linux Completely Fair Scheduler}
\label{sec:org0958630}
\begin{itemize}
\item Highly efficient and scalable fair-share scheduler
\item Aims to spend very little time making choices
\item Important to not waste resources
\begin{itemize}
\item Google used 5\% of CPU time scheduling
\end{itemize}
\item Reducing overhead is a key goal in modern schedulers
\item Goal is to divide the CPU evenly among all competing processes
\item It does so with a virtual runtime (vruntime)
\end{itemize}
\subsubsection{CFS: Basic Operation}
\label{sec:org4ea3699}
\begin{itemize}
\item Each process runs an accumulates vruntime
\item Always picks the process with lowest vruntime
\item CFS varies with sched\textsubscript{latency}
\begin{itemize}
\item Represents the largest time slice size possible
\item Determined by sched\textsubscript{latency}/number of processes
\end{itemize}
\end{itemize}
\subsubsection{Niceness}
\label{sec:org01ec592}
\begin{itemize}
\item Adds weighting to the time slice calculation
\item time slice = Poriton of all processes running * max time slice
\item vruntime = previous vruntime + time just ran * weighting based on niceness
\end{itemize}

\[
\frac{1024}{2048} * \text{sched_latency}
\]
\end{document}
